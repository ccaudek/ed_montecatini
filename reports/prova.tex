% Options for packages loaded elsewhere
\PassOptionsToPackage{unicode}{hyperref}
\PassOptionsToPackage{hyphens}{url}
\PassOptionsToPackage{dvipsnames,svgnames,x11names}{xcolor}
%
\documentclass[
  man]{apa6}
\usepackage{amsmath,amssymb}
\usepackage{lmodern}
\usepackage{iftex}
\ifPDFTeX
  \usepackage[T1]{fontenc}
  \usepackage[utf8]{inputenc}
  \usepackage{textcomp} % provide euro and other symbols
\else % if luatex or xetex
  \usepackage{unicode-math}
  \defaultfontfeatures{Scale=MatchLowercase}
  \defaultfontfeatures[\rmfamily]{Ligatures=TeX,Scale=1}
\fi
% Use upquote if available, for straight quotes in verbatim environments
\IfFileExists{upquote.sty}{\usepackage{upquote}}{}
\IfFileExists{microtype.sty}{% use microtype if available
  \usepackage[]{microtype}
  \UseMicrotypeSet[protrusion]{basicmath} % disable protrusion for tt fonts
}{}
\makeatletter
\@ifundefined{KOMAClassName}{% if non-KOMA class
  \IfFileExists{parskip.sty}{%
    \usepackage{parskip}
  }{% else
    \setlength{\parindent}{0pt}
    \setlength{\parskip}{6pt plus 2pt minus 1pt}}
}{% if KOMA class
  \KOMAoptions{parskip=half}}
\makeatother
\usepackage{xcolor}
\usepackage{graphicx}
\makeatletter
\def\maxwidth{\ifdim\Gin@nat@width>\linewidth\linewidth\else\Gin@nat@width\fi}
\def\maxheight{\ifdim\Gin@nat@height>\textheight\textheight\else\Gin@nat@height\fi}
\makeatother
% Scale images if necessary, so that they will not overflow the page
% margins by default, and it is still possible to overwrite the defaults
% using explicit options in \includegraphics[width, height, ...]{}
\setkeys{Gin}{width=\maxwidth,height=\maxheight,keepaspectratio}
% Set default figure placement to htbp
\makeatletter
\def\fps@figure{htbp}
\makeatother
\setlength{\emergencystretch}{3em} % prevent overfull lines
\providecommand{\tightlist}{%
  \setlength{\itemsep}{0pt}\setlength{\parskip}{0pt}}
\setcounter{secnumdepth}{-\maxdimen} % remove section numbering
% Make \paragraph and \subparagraph free-standing
\ifx\paragraph\undefined\else
  \let\oldparagraph\paragraph
  \renewcommand{\paragraph}[1]{\oldparagraph{#1}\mbox{}}
\fi
\ifx\subparagraph\undefined\else
  \let\oldsubparagraph\subparagraph
  \renewcommand{\subparagraph}[1]{\oldsubparagraph{#1}\mbox{}}
\fi
\newlength{\cslhangindent}
\setlength{\cslhangindent}{1.5em}
\newlength{\csllabelwidth}
\setlength{\csllabelwidth}{3em}
\newlength{\cslentryspacingunit} % times entry-spacing
\setlength{\cslentryspacingunit}{\parskip}
\newenvironment{CSLReferences}[2] % #1 hanging-ident, #2 entry spacing
 {% don't indent paragraphs
  \setlength{\parindent}{0pt}
  % turn on hanging indent if param 1 is 1
  \ifodd #1
  \let\oldpar\par
  \def\par{\hangindent=\cslhangindent\oldpar}
  \fi
  % set entry spacing
  \setlength{\parskip}{#2\cslentryspacingunit}
 }%
 {}
\usepackage{calc}
\newcommand{\CSLBlock}[1]{#1\hfill\break}
\newcommand{\CSLLeftMargin}[1]{\parbox[t]{\csllabelwidth}{#1}}
\newcommand{\CSLRightInline}[1]{\parbox[t]{\linewidth - \csllabelwidth}{#1}\break}
\newcommand{\CSLIndent}[1]{\hspace{\cslhangindent}#1}
\ifLuaTeX
\usepackage[bidi=basic]{babel}
\else
\usepackage[bidi=default]{babel}
\fi
\babelprovide[main,import]{english}
% get rid of language-specific shorthands (see #6817):
\let\LanguageShortHands\languageshorthands
\def\languageshorthands#1{}
% Manuscript styling
\usepackage{upgreek}
\captionsetup{font=singlespacing,justification=justified}

% Table formatting
\usepackage{longtable}
\usepackage{lscape}
% \usepackage[counterclockwise]{rotating}   % Landscape page setup for large tables
\usepackage{multirow}		% Table styling
\usepackage{tabularx}		% Control Column width
\usepackage[flushleft]{threeparttable}	% Allows for three part tables with a specified notes section
\usepackage{threeparttablex}            % Lets threeparttable work with longtable

% Create new environments so endfloat can handle them
% \newenvironment{ltable}
%   {\begin{landscape}\centering\begin{threeparttable}}
%   {\end{threeparttable}\end{landscape}}
\newenvironment{lltable}{\begin{landscape}\centering\begin{ThreePartTable}}{\end{ThreePartTable}\end{landscape}}

% Enables adjusting longtable caption width to table width
% Solution found at http://golatex.de/longtable-mit-caption-so-breit-wie-die-tabelle-t15767.html
\makeatletter
\newcommand\LastLTentrywidth{1em}
\newlength\longtablewidth
\setlength{\longtablewidth}{1in}
\newcommand{\getlongtablewidth}{\begingroup \ifcsname LT@\roman{LT@tables}\endcsname \global\longtablewidth=0pt \renewcommand{\LT@entry}[2]{\global\advance\longtablewidth by ##2\relax\gdef\LastLTentrywidth{##2}}\@nameuse{LT@\roman{LT@tables}} \fi \endgroup}

% \setlength{\parindent}{0.5in}
% \setlength{\parskip}{0pt plus 0pt minus 0pt}

% Overwrite redefinition of paragraph and subparagraph by the default LaTeX template
% See https://github.com/crsh/papaja/issues/292
\makeatletter
\renewcommand{\paragraph}{\@startsection{paragraph}{4}{\parindent}%
  {0\baselineskip \@plus 0.2ex \@minus 0.2ex}%
  {-1em}%
  {\normalfont\normalsize\bfseries\itshape\typesectitle}}

\renewcommand{\subparagraph}[1]{\@startsection{subparagraph}{5}{1em}%
  {0\baselineskip \@plus 0.2ex \@minus 0.2ex}%
  {-\z@\relax}%
  {\normalfont\normalsize\itshape\hspace{\parindent}{#1}\textit{\addperi}}{\relax}}
\makeatother

% \usepackage{etoolbox}
\makeatletter
\patchcmd{\HyOrg@maketitle}
  {\section{\normalfont\normalsize\abstractname}}
  {\section*{\normalfont\normalsize\abstractname}}
  {}{\typeout{Failed to patch abstract.}}
\patchcmd{\HyOrg@maketitle}
  {\section{\protect\normalfont{\@title}}}
  {\section*{\protect\normalfont{\@title}}}
  {}{\typeout{Failed to patch title.}}
\makeatother

\usepackage{xpatch}
\makeatletter
\xapptocmd\appendix
  {\xapptocmd\section
    {\addcontentsline{toc}{section}{\appendixname\ifoneappendix\else~\theappendix\fi\\: #1}}
    {}{\InnerPatchFailed}%
  }
{}{\PatchFailed}
\keywords{keywords\newline\indent Word count: X}
\DeclareDelayedFloatFlavor{ThreePartTable}{table}
\DeclareDelayedFloatFlavor{lltable}{table}
\DeclareDelayedFloatFlavor*{longtable}{table}
\makeatletter
\renewcommand{\efloat@iwrite}[1]{\immediate\expandafter\protected@write\csname efloat@post#1\endcsname{}}
\makeatother
\usepackage{lineno}

\linenumbers
\usepackage{csquotes}
\usepackage{apacite}
\usepackage{amsmath}
\usepackage[T1]{fontenc}
\newcommand\numberthis{\addtocounter{equation}{1}\tag{\theequation}}
\usepackage{setspace}\doublespacing
\usepackage{float}
\floatplacement{figure}{H}
\floatplacement{table}{H}
\usepackage[font=small,skip=12pt]{caption}
\ifLuaTeX
  \usepackage{selnolig}  % disable illegal ligatures
\fi
\IfFileExists{bookmark.sty}{\usepackage{bookmark}}{\usepackage{hyperref}}
\IfFileExists{xurl.sty}{\usepackage{xurl}}{} % add URL line breaks if available
\urlstyle{same} % disable monospaced font for URLs
\hypersetup{
  pdftitle={Anorexia nervosa entails domain specific impairment of adaptive learning under uncertainty},
  pdfauthor={Corrado Caudek1, Ilaria Colpizzi1, \& Claudio Sica2},
  pdflang={en-EN},
  pdfkeywords={keywords},
  colorlinks=true,
  linkcolor={blue},
  filecolor={Maroon},
  citecolor={Blue},
  urlcolor={Blue},
  pdfcreator={LaTeX via pandoc}}

\title{Anorexia nervosa entails domain specific impairment of adaptive learning under uncertainty}
\author{Corrado Caudek\textsuperscript{1}, Ilaria Colpizzi\textsuperscript{1}, \& Claudio Sica\textsuperscript{2}}
\date{}


\shorttitle{Domain specific impairment of learning}

\authornote{

The authors made the following contributions. Corrado Caudek: Conceptualization, Project administration, Formal analysis, Writing - Original Draft Preparation, Writing - Review \& Editing; Ilaria Colpizzi: Writing - Review \& Editing, Software development, Data collection, Data curation; Claudio Sica: Writing - Review \& Editing, Supervision.

Correspondence concerning this article should be addressed to Corrado Caudek, Via di San Salvi n.~12, Complesso di S. Salvi, Padiglione 26, Firenze, 50139, Italy. E-mail: \href{mailto:corrado.caudek@unifi.it}{\nolinkurl{corrado.caudek@unifi.it}}

}

\affiliation{\vspace{0.5cm}\textsuperscript{1} NEUROFARBA Department, Psychology Section, University of Florence, Italy.\\\textsuperscript{2} Health Sciences Department, Psychology Section, University of Florence, Italy.}

\abstract{%
One or two sentences providing a \textbf{basic introduction} to the field, comprehensible to a scientist in any discipline.

Two to three sentences of \textbf{more detailed background}, comprehensible to scientists in related disciplines.

One sentence clearly stating the \textbf{general problem} being addressed by this particular study.

One sentence summarizing the main result (with the words ``\textbf{here we show}'' or their equivalent).

Two or three sentences explaining what the \textbf{main result} reveals in direct comparison to what was thought to be the case previously, or how the main result adds to previous knowledge.

One or two sentences to put the results into a more \textbf{general context}.

Two or three sentences to provide a \textbf{broader perspective}, readily comprehensible to a scientist in any discipline.
}



\begin{document}
\maketitle

\hypertarget{introduction}{%
\section{Introduction}\label{introduction}}

\url{https://doi.org/10.1007/s40167-018-0068-0}

To explore the processes underpinning task performance, computational modeling (i.e., drift diffusion model (DDM) analysis) will be used to explicate the specific processes by means of which domain specificigy influences decision-making (e.g., Golubickis et al.~2017, 2018; Macrae et al.~2017). In any task context, there are two distinct ways in which decisional processing can be biased. These pertain to how a stimulus is processed and how a response is generated, with each source of bias reflecting a different underlying component of decisional processing (Voss et al.~2004, 2013; White and Poldrack 2014). Whereas variability in stimulus processing affects the quality of information gathering during decision-making (i.e., dynamic stimulus bias), adjustments in response preparation influence how much evidence is required before a specific judgment is made (i.e., prior or pre-decisional bias). The theoretical value of a DDM analysis resides in its ability to isolate these independent forms of bias, thereby elucidate the component processes that underpin decision-making (Ratcliff 1978; Ratcliff and Rouder 1998; Ratcliff et al.~2016; Voss et al.~2004, 2013; Wagenmakers 2009).

The DDM assumes that, during binary decision-making (e.g., owned-by-self vs.~owned-by-other), noisy information is continuously sampled until sufficient evidence is acquired to initiate a response (see Fig. 1 for a schematic representation of the model). The duration of the diffusion process is known as the decision time, and the process itself can be characterized by several important parameters. Drift rate (v) estimates the speed of information gathering (i.e., larger drift rate = faster information uptake), thus is interpreted as a measure of the quality of visual processing during decision-making (White and Poldrack 2014). Boundary separa- tion (a) estimates the distance between the two decision thresholds (i.e., self-owned vs.~other-owned), hence indicates how much evidence is required before a response is made (i.e., larger (smaller) values indicate more conservative (liberal) respond- ing). The starting point (z) defines the position between the decision thresholds at which evidence accumulation begins. If z is not centered between the thresholds (i.e., z = 0.5), this denotes an a priori bias in favor of the response that is closer to the starting point (White and Poldrack 2014). In other words, less evidence is required to reach the preferred (vs.~non-preferred) threshold. Finally, the duration of all non-decisional processes is given by the additional parameter t0, which is taken to indicate biases in stimulus encoding and response execution (Voss et al.~2010).

\hypertarget{methods}{%
\section{Methods}\label{methods}}

We report how we determined our sample size, all data exclusions (if any), all manipulations, and all measures in the study.

\hypertarget{participants}{%
\subsection{Participants}\label{participants}}

\hypertarget{material}{%
\subsection{Material}\label{material}}

\hypertarget{procedure}{%
\subsection{Procedure}\label{procedure}}

\hypertarget{data-analysis}{%
\subsection{Data analysis}\label{data-analysis}}

We used R (Version 4.2.0; R Core Team, 2020) and the R-packages \emph{bayesplot} (Version 1.9.0; Gabry et al., 2019), \emph{brms} (Version 2.17.0; Bürkner, 2017, 2018), \emph{corrplot2021} (\textbf{R-corrplot2021?}), \emph{dplyr} (Version 1.0.9; Wickham et al., 2020), \emph{forcats} (Version 0.5.1; Wickham, 2020a), \emph{ggplot2} (Version 3.3.6; Wickham, 2016), \emph{ggthemes} (Version 4.2.4; Arnold, 2019), \emph{glue} (Version 1.6.2; Hester, 2020), \emph{gt} (Version 0.6.0; Arnold, 2019; Iannone et al., 2021), \emph{here} (Version 1.0.1; Müller, 2020), \emph{kableExtra} (Version 1.3.4; Zhu, 2021), \emph{khroma} (Version 1.8.0; \textbf{R-khroma?}), \emph{knitr} (Version 1.39; Xie, 2015), \emph{lavaan} (Version 0.6.11; Rosseel, 2012), \emph{papaja} (Version 0.1.0.9999; Aust \& Barth, 2020), \emph{patchwork} (Version 1.1.1; Pedersen, 2020), \emph{projpred} (Version 2.1.2; Piironen et al., 2020), \emph{purrr} (Version 0.3.4; Henry \& Wickham, 2020), \emph{Rcpp} (Eddelbuettel \& Balamuta, 2017; Version 1.0.8.3; Eddelbuettel \& François, 2011), \emph{readr} (Version 2.1.2; Wickham \& Hester, 2020), \emph{rio} (Version 0.5.29; Chan et al., 2018), \emph{semPlot} (Version 1.1.5; Epskamp, 2019), \emph{stringr} (Version 1.4.0; Wickham, 2019), \emph{tibble} (Version 3.1.7; Müller \& Wickham, 2020), \emph{tidyr} (Version 1.2.0; Wickham, 2020b), \emph{tidyverse} (Version 1.3.1; Wickham et al., 2019), \emph{tinylabels} (Version 0.2.3; Barth, 2021), \emph{viridis} (Version 0.6.2; Garnier, 2018a, 2018b), and \emph{viridisLite} (Version 0.4.0; Garnier, 2018b) for all our analyses.

\hypertarget{results}{%
\section{Results}\label{results}}

\hypertarget{discussion}{%
\section{Discussion}\label{discussion}}

\newpage

\hypertarget{references}{%
\section{References}\label{references}}

\hypertarget{refs}{}
\begin{CSLReferences}{1}{0}
\leavevmode\vadjust pre{\hypertarget{ref-R-ggthemes}{}}%
Arnold, J. B. (2019). \emph{Ggthemes: Extra themes, scales and geoms for 'ggplot2'}. \url{https://CRAN.R-project.org/package=ggthemes}

\leavevmode\vadjust pre{\hypertarget{ref-R-papaja}{}}%
Aust, F., \& Barth, M. (2020). \emph{{papaja}: {Create} {APA} manuscripts with {R Markdown}}. \url{https://github.com/crsh/papaja}

\leavevmode\vadjust pre{\hypertarget{ref-R-tinylabels}{}}%
Barth, M. (2021). \emph{{tinylabels}: Lightweight variable labels}. \url{https://github.com/mariusbarth/tinylabels}

\leavevmode\vadjust pre{\hypertarget{ref-R-brms_a}{}}%
Bürkner, P.-C. (2017). {brms}: An {R} package for {Bayesian} multilevel models using {Stan}. \emph{Journal of Statistical Software}, \emph{80}(1), 1--28. \url{https://doi.org/10.18637/jss.v080.i01}

\leavevmode\vadjust pre{\hypertarget{ref-R-brms_b}{}}%
Bürkner, P.-C. (2018). Advanced {Bayesian} multilevel modeling with the {R} package {brms}. \emph{The R Journal}, \emph{10}(1), 395--411. \url{https://doi.org/10.32614/RJ-2018-017}

\leavevmode\vadjust pre{\hypertarget{ref-R-rio}{}}%
Chan, C., Chan, G. C., Leeper, T. J., \& Becker, J. (2018). \emph{Rio: A swiss-army knife for data file i/o}.

\leavevmode\vadjust pre{\hypertarget{ref-R-Rcpp_b}{}}%
Eddelbuettel, D., \& Balamuta, J. J. (2017). {Extending extit{R} with extit{C++}: A Brief Introduction to extit{Rcpp}}. \emph{PeerJ Preprints}, \emph{5}, e3188v1. \url{https://doi.org/10.7287/peerj.preprints.3188v1}

\leavevmode\vadjust pre{\hypertarget{ref-R-Rcpp_a}{}}%
Eddelbuettel, D., \& François, R. (2011). {Rcpp}: Seamless {R} and {C++} integration. \emph{Journal of Statistical Software}, \emph{40}(8), 1--18. \url{https://doi.org/10.18637/jss.v040.i08}

\leavevmode\vadjust pre{\hypertarget{ref-R-semPlot}{}}%
Epskamp, S. (2019). \emph{semPlot: Path diagrams and visual analysis of various SEM packages' output}. \url{https://CRAN.R-project.org/package=semPlot}

\leavevmode\vadjust pre{\hypertarget{ref-R-bayesplot}{}}%
Gabry, J., Simpson, D., Vehtari, A., Betancourt, M., \& Gelman, A. (2019). Visualization in bayesian workflow. \emph{J. R. Stat. Soc. A}, \emph{182}, 389--402. \url{https://doi.org/10.1111/rssa.12378}

\leavevmode\vadjust pre{\hypertarget{ref-R-viridis}{}}%
Garnier, S. (2018a). \emph{Viridis: Default color maps from 'matplotlib'}. \url{https://CRAN.R-project.org/package=viridis}

\leavevmode\vadjust pre{\hypertarget{ref-R-viridisLite}{}}%
Garnier, S. (2018b). \emph{viridisLite: Default color maps from 'matplotlib' (lite version)}. \url{https://CRAN.R-project.org/package=viridisLite}

\leavevmode\vadjust pre{\hypertarget{ref-R-purrr}{}}%
Henry, L., \& Wickham, H. (2020). \emph{Purrr: Functional programming tools}. \url{https://CRAN.R-project.org/package=purrr}

\leavevmode\vadjust pre{\hypertarget{ref-R-glue}{}}%
Hester, J. (2020). \emph{Glue: Interpreted string literals}. \url{https://CRAN.R-project.org/package=glue}

\leavevmode\vadjust pre{\hypertarget{ref-R-gt}{}}%
Iannone, R., Cheng, J., \& Schloerke, B. (2021). \emph{Gt: Easily create presentation-ready display tables}.

\leavevmode\vadjust pre{\hypertarget{ref-R-here}{}}%
Müller, K. (2020). \emph{Here: A simpler way to find your files}. \url{https://CRAN.R-project.org/package=here}

\leavevmode\vadjust pre{\hypertarget{ref-R-tibble}{}}%
Müller, K., \& Wickham, H. (2020). \emph{Tibble: Simple data frames}. \url{https://CRAN.R-project.org/package=tibble}

\leavevmode\vadjust pre{\hypertarget{ref-R-patchwork}{}}%
Pedersen, T. L. (2020). \emph{Patchwork: The composer of plots}. \url{https://CRAN.R-project.org/package=patchwork}

\leavevmode\vadjust pre{\hypertarget{ref-R-projpred}{}}%
Piironen, J., Paasiniemi, M., Catalina, A., \& Vehtari, A. (2020). \emph{Projpred: Projection predictive feature selection}. \url{https://CRAN.R-project.org/package=projpred}

\leavevmode\vadjust pre{\hypertarget{ref-R-base}{}}%
R Core Team. (2020). \emph{R: A language and environment for statistical computing}. R Foundation for Statistical Computing. \url{https://www.R-project.org/}

\leavevmode\vadjust pre{\hypertarget{ref-R-lavaan}{}}%
Rosseel, Y. (2012). {lavaan}: An {R} package for structural equation modeling. \emph{Journal of Statistical Software}, \emph{48}(2), 1--36. \url{http://www.jstatsoft.org/v48/i02/}

\leavevmode\vadjust pre{\hypertarget{ref-R-ggplot2}{}}%
Wickham, H. (2016). \emph{ggplot2: Elegant graphics for data analysis}. Springer-Verlag New York. \url{https://ggplot2.tidyverse.org}

\leavevmode\vadjust pre{\hypertarget{ref-R-stringr}{}}%
Wickham, H. (2019). \emph{Stringr: Simple, consistent wrappers for common string operations}. \url{https://CRAN.R-project.org/package=stringr}

\leavevmode\vadjust pre{\hypertarget{ref-R-forcats}{}}%
Wickham, H. (2020a). \emph{Forcats: Tools for working with categorical variables (factors)}. \url{https://CRAN.R-project.org/package=forcats}

\leavevmode\vadjust pre{\hypertarget{ref-R-tidyr}{}}%
Wickham, H. (2020b). \emph{Tidyr: Tidy messy data}. \url{https://CRAN.R-project.org/package=tidyr}

\leavevmode\vadjust pre{\hypertarget{ref-R-tidyverse}{}}%
Wickham, H., Averick, M., Bryan, J., Chang, W., McGowan, L. D., François, R., Grolemund, G., Hayes, A., Henry, L., Hester, J., Kuhn, M., Pedersen, T. L., Miller, E., Bache, S. M., Müller, K., Ooms, J., Robinson, D., Seidel, D. P., Spinu, V., \ldots{} Yutani, H. (2019). Welcome to the {tidyverse}. \emph{Journal of Open Source Software}, \emph{4}(43), 1686. \url{https://doi.org/10.21105/joss.01686}

\leavevmode\vadjust pre{\hypertarget{ref-R-dplyr}{}}%
Wickham, H., François, R., Henry, L., \& Müller, K. (2020). \emph{Dplyr: A grammar of data manipulation}. \url{https://CRAN.R-project.org/package=dplyr}

\leavevmode\vadjust pre{\hypertarget{ref-R-readr}{}}%
Wickham, H., \& Hester, J. (2020). \emph{Readr: Read rectangular text data}. \url{https://CRAN.R-project.org/package=readr}

\leavevmode\vadjust pre{\hypertarget{ref-R-knitr}{}}%
Xie, Y. (2015). \emph{Dynamic documents with {R} and knitr} (2nd ed.). Chapman; Hall/CRC. \url{https://yihui.org/knitr/}

\leavevmode\vadjust pre{\hypertarget{ref-R-kableExtra}{}}%
Zhu, H. (2021). \emph{kableExtra: Construct complex table with 'kable' and pipe syntax}. \url{https://CRAN.R-project.org/package=kableExtra}

\end{CSLReferences}


\clearpage
\makeatletter
\efloat@restorefloats
\makeatother


\begin{appendix}
\hypertarget{covid-19-news-data-set-full-model-with-13-covariates}{%
\section{COVID-19 news data set: full model with 13
covariates}\label{covid-19-news-data-set-full-model-with-13-covariates}}

\begin{table}[H]

\begin{center}
\begin{threeparttable}

\caption{\label{tab:table13covid}Posterior mean, standard error, 95\% credible interval and $\hat{R}$
    statistic for each parameter of the full model (13 covariates) predicting COVID-19 news truth discernment.}

\small{

\begin{tabular}{cccccc}
\toprule
parameter & \multicolumn{1}{c}{mean} & \multicolumn{1}{c}{SE} & \multicolumn{1}{c}{lower bound} & \multicolumn{1}{c}{upper bound} & \multicolumn{1}{c}{Rhat}\\
\midrule
Intercept & -0.331 & 0.050 & -0.429 & -0.231 & 1.000\\
age & 0.166 & 0.036 & 0.094 & 0.236 & 1.000\\
agsu & 0.004 & 0.048 & -0.090 & 0.098 & 1.000\\
conv & -0.125 & 0.054 & -0.230 & -0.020 & 1.000\\
miis & -0.126 & 0.035 & -0.195 & -0.057 & 1.000\\
rfsp & -0.033 & 0.047 & -0.126 & 0.060 & 1.000\\
rfsn & -0.026 & 0.042 & -0.108 & 0.056 & 1.000\\
educ & 0.073 & 0.030 & 0.013 & 0.132 & 1.000\\
poli & 0.056 & 0.033 & -0.009 & 0.121 & 1.000\\
spir & 0.107 & 0.044 & 0.021 & 0.192 & 1.000\\
para & -0.069 & 0.039 & -0.145 & 0.007 & 1.000\\
crit & 0.157 & 0.029 & 0.100 & 0.215 & 1.000\\
cosp & -0.253 & 0.035 & -0.322 & -0.184 & 1.000\\
supe & -0.069 & 0.040 & -0.147 & 0.010 & 1.000\\
simp & -0.081 & 0.035 & -0.150 & -0.012 & 1.000\\
conf & 0.043 & 0.031 & -0.018 & 0.103 & 1.000\\
rese & -0.016 & 0.033 & -0.082 & 0.048 & 1.000\\
sex & 0.125 & 0.067 & -0.007 & 0.256 & 1.000\\
sigma & 0.749 & 0.020 & 0.711 & 0.790 & 1.000\\
\bottomrule
\end{tabular}

}

\end{threeparttable}
\end{center}

\end{table}

\newpage

\hypertarget{covid-19-news-data-set-best-projection-model-ignoring-the-political-news-data}{%
\section{COVID-19 news data set: best projection model (ignoring the
political news
data)}\label{covid-19-news-data-set-best-projection-model-ignoring-the-political-news-data}}

\begin{table}[H]

\begin{center}
\begin{threeparttable}

\caption{\label{tab:tablebestsubcovid}Posterior mean, standard error, 95\% credible interval and $\hat{R}$
    statistic for each parameter of the best projection model predicting COVID-19 news truth discernment.}

\small{

\begin{tabular}{cccccc}
\toprule
parameter & \multicolumn{1}{c}{mean} & \multicolumn{1}{c}{SE} & \multicolumn{1}{c}{lower bound} & \multicolumn{1}{c}{upper bound} & \multicolumn{1}{c}{Rhat}\\
\midrule
Intercept & -0.368 & 0.043 & -0.453 & -0.283 & 1.000\\
miis & -0.142 & 0.034 & -0.209 & -0.075 & 1.000\\
cosp & -0.285 & 0.033 & -0.350 & -0.221 & 1.000\\
crit & 0.172 & 0.029 & 0.116 & 0.230 & 1.000\\
age & 0.188 & 0.032 & 0.126 & 0.250 & 1.000\\
supe & -0.122 & 0.036 & -0.193 & -0.052 & 1.000\\
conv & -0.142 & 0.048 & -0.235 & -0.049 & 1.000\\
sigma & 0.757 & 0.020 & 0.718 & 0.798 & 1.000\\
\bottomrule
\end{tabular}

}

\end{threeparttable}
\end{center}

\end{table}
\end{appendix}

\end{document}
